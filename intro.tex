Foliations have had famous applications in the realm of real differential
geometry, due to the work of Thurston and his geometrization conjecture.
However, holomorphic foliations in complex geometry have also been a fruitful
area of study. Here we will specifically look at the applications of foliations
to the study of complex algebraic surfaces, and their birational geometry. In
analogy with the Enriques classification of surfaces and the Minimal Model
Program, foliations of surfaces also have a rich classification theory
\cite{brunella_book}. This report is an introduction to the basic concepts of
the theory of foliations on algebraic surfaces, referencing important results
but with many omissions, including almost all of the classification theory. We
have attempted to motivate the concepts with lots of basic examples, and
judicious use of pictures.

The structure of the report is as follows. We begin with an introduction to the
basic concepts of foliations in \cref{sec:foliations}, and the local analysis of
singularities, mentioning the Camacho--Sad separatrix theorem, and the behaviour
of foliations under blowup including Seidenberg's theorem. \cref{sec:curves}
covers index theorems giving some of the intersection theory of the canonical
bundle of a foliation, which we then apply in \cref{sec:fibrations} to some
examples of foliations arising from fibrations. Finally, in \cref{sec:stability}
we make some remarks about the work of Bogomolov and McQuillan towards the
Green--Griffiths conjecture, before going on a tenuously related tangent about
slope stability and the Donaldson--Uhlenbeck--Yau theorem.
