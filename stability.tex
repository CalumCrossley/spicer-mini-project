
We introduce the Green--Griffiths conjecture, and sketch some of the ideas due
to Bogomolov and McQuillan applying the theory of foliations to the problem.
Motivated by an example of these ideas, we digress into a discussion of some
basic concepts of slope stability for sheaves.

\subsection{The Green--Griffiths conjecture}

The Green--Griffiths conjecture concerns the transcendental geometry of general
type varieties.

\begin{conjecture}[Green--Griffiths, \cite{griffiths_80}]
    Let $X$ be a smooth complex projective variety of general type. Then $X$
    admits no Zariski dense entire holomorphic curves.
\end{conjecture}

There is a more general form of the conjecture, as follows:

\begin{conjecture}[Green--Griffiths--Lang, \cite{lang_86}]
    Let $X$ be a smooth complex projective variety of general type. There is a
    proper algebraic subvariety $Y\subset X$ such that every entire holomorphic
    curve $f:\C\to X$ factors through $Y$.
\end{conjecture}

We are considering a holomorphic map $f:\C\to X$, which induces also a map
$df:\C\to\P(T_X)$. Writing $\O(-1)$ for the relative tautological bundle on
$\P(T_X)$, we have (tautologically) $(df)^*\O(-1)=T_C$. Now suppose $\O(1)$ is
big. This assumption is intended as an approximation of the condition of being
general type. We may find an effective divisor $D\ge0$ in $\P(T_X)$ representing
$\O(1)$, which is locally a section of the bundle over $X$, although possibly
not globally. Up to finite degree issues, this gives a rational section of
$\P(T_X)$, which as mentioned earlier is precisely the data of a foliation on
$X$, call it $\scrF$.

Now from $(df)^*\O(-1)=T_C$ we get $(df)^*\O(1)=K_C=2g(C)-2$, so if $C$ is
rational $C\cdot D<0$. But this forces $C$ to be a component of $D$, and so in
fact $\scrF$ is tangent to $C$. In other words, from the assumption that $\O(1)$
is big we have produced a foliation on $X$ which is tangent to every rational
curve in $X$. The existence of such a foliation is a strong constraint, and this
argument is part of work due to Bogomolov and McQuillan \cite{mcquillan_98}
which lead to certain restricted results related to the conjecture.

\subsection{Ample vector bundles}

We would like to better understand this condition of $\O_{\P(T_X)}(1)$ being
big in terms of $X$ and $T_X$ directly. It turns out to be equivalent to a
notion of ``ample'' for the vector bundle $T_X$.

\begin{proposition}[\cite{hartshorne_66}, 3.2] \label{prop:ample}
    Let $\calE$ be a vector bundle on a variety $X$. Then $\calE$ is an ample
    vector bundle on $X$ iff the tautological line bundle
    $\O_{\P(\calE^\vee)}(1)$ on $\P(\calE^\vee)$ is ample.
\end{proposition}

We recall from \cite{hartshorne_66} the notion of an ample vector bundle:

\begin{definition}
    We say a vector bundle $\calE$ over a variety $X$ is \emph{ample} if for any
    coherent sheaf $\calF$ the sheaf $\calF\otimes\Sym^n\calE$ is generated by
    global sections for $n\gg0$.
\end{definition}

\begin{remark}
    When $\calE=\calL$ is a line bundle we have $\Sym^n\calL=\calL^{\otimes n}$,
    so this recovers the definition of ample for line bundles.
\end{remark}

\begin{remark}
    Motivated by the analogous statement for line bundles, one might ask whether
    this definition is equivalent to requiring $\Sym^n\calE$ to give an
    embedding into the corresponding Grassmannian for $n\gg0$. In fact this is
    not true: take for example $\O\oplus\O(1)$ on $\P^1$. This is not ample,
    since $\O(-1)\otimes\Sym^n(\O\oplus\O(1))$ has a summand of $\O(-1)$ for all
    $n\ge0$. However, the complete system $(1\oplus0,0\oplus x,0\oplus y)$
    induces an embedding $\P^1\hookrightarrow\Gr(2,\C^3)$;
    $[x:y]\mapsto\langle(1,0,0),(0,x,y)\rangle$.
\end{remark}

The key point to proving \cref{prop:ample} is the following fact:

\begin{lemma}
    If $p:\P(\calE^\vee)\to X$ is the projection map, then for every coherent
    sheaf $\calF$ on $X$ we have
    \begin{equation*}
        \dR p_*(p^*\calF\otimes\O_{\P(\calE^\vee)}(n))
            = \calF\otimes\Sym^n(\calE).
    \end{equation*}
\end{lemma}

\begin{proof}
    By the projection formula we have
    \begin{equation*}
        \dR p_*(p^*\calF\otimes\O_{\P(\calE^\vee)}(n))
            = \calF\otimes^\dL\dR p_*\O_{\P(\calE^\vee)}(n),
    \end{equation*}
    and $\dR p_*\O_{\P(\calE^\vee)}(n)=\Sym^n\calE$ as a relative version of
    \begin{equation*}
        H^k(\O_{\P(V)}(n)) = \begin{dcases*}
            \Sym^nV^\vee & $k=0$, $n\ge0$, \\
            0 & otherwise.
        \end{dcases*}
    \end{equation*}
\end{proof}

Motivated by the construction in the previous section, suppose we want to
construct a general type surface $X$ such that $\O_{\P(T_X)}(1)$ is big. From
\cref{prop:ample}, this is equivalent to requiring that $T_X$ is ample. We will
consider a degree $d$ hypersurface in $\P^3$. There is the normal bundle exact
sequence
\begin{equation*}
    0 \to N_{X/\P^3}^\vee \to \Omega^1_{\P^3}|_X \to \Omega^1_X \to 0,
\end{equation*}
and $N_{X/\P^3}=\O(-d)$ by adjunction. Taking determinants we see that
$\O(-4)=K_{\P^3}=K_X\otimes\O(-d)$, so at least $\det\Omega^1_X=K_X$ is ample
for $d\gg0$. It is not true that a vector bundle with ample determinant bundle
is always itself ample (although the converse holds). However, in this
particular case we can try to apply the following result:

\begin{theorem}[\cite{fulger_22}, 3.10]\label{thm:stable ample}
    If $\calE$ is a slope semistable vector bundle, and the discriminant
    \begin{equation*}
        \Delta(\calE)
            = 2\rank\calE\cdot c_2(\calE) - (\rank\calE-1)c_1(\calE)^2
    \end{equation*}
    is zero, then $\calE$ is ample iff $\det\calE$ is ample.
\end{theorem}

We will see in the next section that $\Omega^1_X$ is slope semistable (in fact
slope stable), after first introducing the notion of slope stability. However,
in computing the discriminant we find
\begin{align*}
    \Delta(\Omega^1_X)
        &= 4c_2(X) - c_1(X)^2 \\
        &= 4\chi(X) - K_X\cdot K_X \\
        &= 4\chi(X) - d(d-4)^2.
\end{align*}
In fact $\chi(X)=d(d^2-4d+6)$ \cite{maxim_24}, giving
$\Delta(\Omega^1_X)=3d^2-8d+8$. This is positive for all $d$, and so
unfortunately the theorem doesn't apply in our example.

\begin{example}
    For an example where $\det\calE$ is ample but $\calE$ is \emph{not} ample,
    consider $\calE=\O(-1)\oplus\O(2)$ on $\P^1$. Then $\det\calE=\O(1)$ is
    clearly ample, but $\calE$ cannot be ample as the summand $\O(-1)$ is not
    ample.
\end{example}

\begin{remark}
    To see that $\det\calE$ is ample whenever $\calE$ is ample, note that
    $\Sym^n\calE$ is ample and generated by global sections for $n\gg0$.
    Tensoring with a globally generated sheaf preserves ampleness, so
    $(\Sym^n\calE)^{\otimes r}$ is ample, and hence the quotient
    $\wedge^r\Sym^n\calE=\det(\Sym^n\calE)$ is also ample. But this line bundle
    is acted on by $\GL(\calE)$, and the only characters of $\GL$ are powers of
    the determinant, so $\det(\Sym^n\calE)=(\det\calE)^{\otimes m}$ for some
    $m$. Hence $\det\calE$ is ample.
\end{remark}

\subsection{Slope stability}

The notion of slope stability is motivated by the application of Mumford's
geometric invariant theory in the construction of moduli spaces of sheaves; it
gives restricted classes of sheaves for which we can construct well-behaved
moduli spaces via Grothendieck's Quot construction.

\begin{definition}
    Suppose $X$ is a complex projective variety, with ample line bundle $\O(1)$.
    For a torsion-free coherent sheaf $\calE$ on $X$, we define the \emph{slope}
    of $\calE$ to be
    \begin{equation*}
        \mu(\calE) = \deg\calE / \rank\calE,
    \end{equation*}
    where $\deg\calE/(\dim X)!$ is the coefficient of $m^{\dim X}$ in the
    Hilbert polynomial $m\mapsto\chi(\calE\otimes\O(m))$. We say $\calE$ is
    slope \emph{semistable} if for any subsheaf $\calF\subset\calE$ with
    $0<\rank\calF<\rank\calE$ we have $\mu(\calF)\le\mu(\calE)$, and
    \emph{stable} if the inequality is strict. A direct sum of stable sheaves of
    the same slope is called \emph{polystable}.
\end{definition}

We will only be considering slope stability, and so we drop the slope prefix and
write simply ``$\calE$ is (semi/poly)stable''. 

\begin{remark}
    Line bundles are trivially stable, having no subsheaves of smaller rank.
\end{remark}

\subsubsection{Relation to ample vector bundles}

To apply \cref{prop:ample} in the situation of $\Omega^1_X$ above, we appeal to
two major theorems:

\begin{theorem}[Aubin \cite{aubin_82}, Yau \cite{yau_78}]
    For a general type smooth complex projective variety, the tangent bundle
    admits a K\"ahler--Einstein metric.
\end{theorem}

\begin{theorem}[\cite{kobayashi_book}, \cite{lubke_83}]
    If the tangent bundle of a smooth complex projective variety admits a
    K\"ahler--Einstein metric, then it is stable.
\end{theorem}

The stability of a vector bundle is equivalent to stability of its dual
(\cref{prop:stable dual}), so these results imply stability of $\Omega^1_X$ in
our example earlier.

\subsubsection{The Donaldson--Uhlenbeck--Yau Theorem}

A key theorem in the theory of slope stability is the following theorem,
formerly known as the Calabi conjecture.

\begin{theorem}[Donaldson--Uhlenbeck--Yau]
    If $X$ is a smooth complex projective manifold, then a holomorphic vector
    bundle $\calE$ on $X$ is polystable iff it admits a Hermite--Einstein
    metric.
\end{theorem}

\begin{definition}
    On a compact K\"ahler manifold $(X,\omega)$ with a holomorphic vector bundle
    $\calE$, a Hermitian metric $h$ on $\calE$ is \emph{Hermite--Einstein} if
    its Chern connection $A$ has curvature $F_A$ satisfying
    \begin{equation*}
        F_A\wedge\omega^{\dim X-1}
            = \lambda\cdot(\id_\calE\otimes\omega^{\dim X})
    \end{equation*}
    for some $\lambda\in\C$.
\end{definition}

In fact the constant $\lambda$ is geometrically constrained:
\begin{align*}
    \deg\calE
        &= \int_Xc_1(\calE)\wedge\omega^{\dim X-1} \\
        &= \frac{i}{2\pi}\int_X\tr(F_A)\wedge\omega^{\dim X-1} \\
        &= \frac{i}{2\pi}\lambda\rank\calE\vol(X),
\end{align*}
so $\lambda=-2\pi i\mu(\calE)/\vol(X)$, which indicates how slope stability
might be related. Indeed, the existence of a Hermite--Einstein metric given
polystability is the hard part of the theorem; the other direction is relatively
easy.

We now turn to look at some basic properties of stability.

\begin{proposition}\label{prop:line}
    A torsion-free coherent sheaf $\calE$ is stable iff $\calE\otimes\calL$ is
    stable for any line bundle $\calL$.
\end{proposition}

\begin{proof}
    Any subsheaf of $\calE\otimes\calL$ is of the form $\calF\otimes\calL$ for
    a subsheaf $\calF$ of $\calE$ since $-\otimes\calL$ is an exact equivalence,
    so this follows from the formula
    \begin{equation*}
        \mu(\calF\otimes\calL) = \mu(\calF) + \deg(\calL).
    \end{equation*}
\end{proof}

\begin{proposition}\label{prop:stable dual}
    A vector bundle $\calE$ is stable iff $\calE^\vee$ is stable.
\end{proposition}

\begin{proof}
    Since $\calE=\calE^{\vee\vee}$, it suffices to prove the forward
    implication. Now $\calE^\vee=\calE\otimes(\det\calE)^\vee$, since a volume
    form defines a perfect pairing $\calE\otimes\calE\to\C$, so the result
    follows from \cref{prop:line}.
\end{proof}

\begin{proposition}\label{prop:slope SES}
    If
    \begin{equation*}
        0\to\calE'\to\calE\to\calE''\to0
    \end{equation*}
    is a short exact sequence of torsion-free coherent sheaves, then
    \begin{equation*}
        \min\{\mu(\calE'),\mu(\calE'')\}
            \le \mu(\calE) \le \max\{\mu(\calE'),\mu(\calE'')\},
    \end{equation*}
    with equality at either end iff $\mu(\calE')=\mu(\calE)=\mu(\calE'')$ or one
    of $\calE'$ and $\calE''$ vanishes.
\end{proposition}

\begin{proof}
    Recall that degree and rank are additive in exact sequences. Hence
    \begin{align*}
        \mu(\calE)
            = \frac{\deg\calE'+\deg\calE''}{\rank\calE'+\rank\calE''}
            = \frac{\rank\calE'}{\rank\calE'+\rank\calE''}\cdot\mu(\calE')
            + \frac{\rank\calE''}{\rank\calE'+\rank\calE''}\cdot\mu(\calE''),
    \end{align*}
    which is a positive weighted average of total weight 1. If the endpoints of
    the average are distinct, we can only have an equality if one of the weights
    vanishes.
\end{proof}

\begin{proposition}\label{prop:duality}
    A torsion-free coherent sheaf $\calE$ is semistable (resp. stable) iff for
    all quotients $\calG$ of $\calE$ with $0<\rank\calG<\rank\calE$ we have
    $\mu(\calG)\ge\mu(\calE)$ (resp. $\mu(\calG)>\mu(\calE)$).
\end{proposition}

\begin{proof}
    This is immediate from \cref{prop:slope SES}.
\end{proof}

\begin{proposition}
    If
    \begin{equation*}
        0\to\calE'\to\calE\to\calE''\to0
    \end{equation*}
    is a short exact sequence of nonzero torsion-free coherent sheaves with
    $\mu(\calE')=\mu(\calE)=\mu(\calE'')$, then $\calE$ is semistable iff
    $\calE'$ and $\calE''$ are both semistable.
\end{proposition}

\begin{proof}
    If $\calE$ is semistable then $\calE'$ is semistable by transitivity of
    subsheaves, and similarly for $\calE''$ using \cref{prop:duality}. For the
    converse, assume $\calE'$ and $\calE''$ are semistable, and $\calF$ is a
    subsheaf of $\calE$. We have
    \begin{equation*}
        \begin{tikzcd}
            0 \ar[r] & \calE' \ar[r] & \calE \ar[r] & \calE'' \ar[r] & 0 \\
            0 \ar[r] & \calF' \ar[r] \ar[u,hook] &
                \calF \ar[r] \ar[u,hook] & \calF'' \ar[r] \ar[u,hook] & 0,
        \end{tikzcd}
    \end{equation*}
    where $\calF'=\calE'\cap\calF$. If $\calF'=0$ then
    $\mu(\calF)=\mu(\calF'')\le\mu(\calE'')=\mu(\calE)$, and similarly for
    $\calF''=0$. Otherwise $\mu(\calF')\le\mu(\calE')=\mu(\calE)$, and similarly
    for $\calF''$, so $\mu(\calF)\le\mu(\calE)$ by \cref{prop:slope SES}.
\end{proof}

\begin{corollary}
    If $\calE,\calE'$ are semistable with the same slope, then
    $\calE\oplus\calE'$ is also semistable with that same slope. Hence
    polystable implies semistable.
\end{corollary}

\begin{remark}
    If $\mu(\calE)\ne\mu(\calE')$, then $\calE\oplus\calE'$ is not semistable in
    general. For example, consider $\O\oplus\O(1)$ on $\P^1$. We have
    $\mu(\O\oplus\O(1))=1/2$, while $\mu(\O(1))=1$.
\end{remark}

We conclude by noting the following innocuous-looking fact.

\begin{theorem}[Narasimhan--Sedhadri \cite{narasimhan_65}]
    If $\calE,\calE'$ are semistable vector bundles, then so is
    $\calE\otimes\calE'$.
\end{theorem}

This fact can be seen from the Donaldson--Uhlenbeck--Yau Theorem, because tensor
products of Hermite--Einstein metrics are again Hermite--Einstein; for
$h=h_1\otimes h_2$ we have $F_A=F_{A_1}\otimes1+1\otimes F_{A_2}$. For a long
time there have been no elementary approaches found for proving this theorem,
and it is worth noting that the statement fails in positive characteristic even
for curves. See \cite{maculan_18} for a nice exposition on the context of this
tensor product theorem, and the relation to representation theory.
