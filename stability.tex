
\subsection{The Green-Griffiths conjecture}

% TODO: intro / context

\begin{conjecture}[Green--Griffiths, \cite{griffiths_80}]
    Let $X$ be a smooth complex variety of general type. Then $X$ admits no
    Zariski dense holomorphic curves.
\end{conjecture}

Blah ideas due to Bogomolov, McQuillan. We consider a holomorphic map
$f:\C\to X$, which induces also a map $df:\C\to\P(T_X)$. Writing $\O(-1)$ for
the relative tautological bundle on $\P(T_X)$, we have (tautologically)
$(df)^*\O(-1)=T_C$. Now suppose $\O(1)$ is big. In fact this is equivalent to
the vector bundle $T_X$ being ample in a suitable sense which we will look at
later. % TODO: T_X vs Omega^1_X stuff % TODO: bigness? (just say ample?)
This assumption is intended as an approximation of the condition of being
general type. We may find an effective divisor $D\ge0$ in $\P(T_X)$ representing
$\O(1)$, which is locally a section of the bundle over $X$, although possibly
not globally. For simplicity, we assume that it is a section, which can be
ensured by a suitable base-change. Then we have a section of $\P(T_X)$, which as
mentioned earlier is precisely a foliation, say $\scrF$, on $X$.

Now from $(df)^*\O(-1)=T_C$ we get $(df)^*\O(1)=K_C=2g(C)-2$, so if $C$ is
rational $C\cdot D<0$. But this forces $C$ to be a component of $D$, and so in
fact $\scrF$ is tangent to $C$. % why?
In other words, from the assumption that $\O(1)$ is big we have produced a
foliation on $X$ which is tangent to every rational curve in $X$. The existence
of such a foliation is a strong constraint, and this argument is part of work
due to Bogomolov and McQuillan \cite{mcquillan_98} which lead to certain
restricted results related to the conjecture.

% TODO: hypersurface example motivation for ampleness and stability section

\subsection{Ampleness and slope stability}

% TODO: definitions, tensors and sums, K-E
