
% TODO: in this section we blah

\subsection{The Green--Griffiths conjecture}

The Green--Griffiths conjecture concerns the transcendental geometry of general
type varieties.

\begin{conjecture}[Green--Griffiths, \cite{griffiths_80}]
    Let $X$ be a smooth complex projective variety of general type. Then $X$
    admits no Zariski dense entire holomorphic curves.
\end{conjecture}

There is a more general form of the conjecture, as follows:

\begin{conjecture}[Green--Griffiths--Lang, \cite{lang_86}]
    Let $X$ be a smooth complex projective variety of general type. There is a
    proper algebraic subvariety $Y\subset X$ such that every entire holomorphic
    curve $f:\C\to X$ factors through $Y$.
\end{conjecture}

Part of the motivating work involving foliations in algebraic geometry is an
approach to the conjecture due to Bogomolov and McQuillan. We are considering a
holomorphic map $f:\C\to X$, which induces also a map $df:\C\to\P(T_X)$. Writing
$\O(-1)$ for the relative tautological bundle on $\P(T_X)$, we have
(tautologically) $(df)^*\O(-1)=T_C$. Now suppose $\O(1)$ is big. This assumption
is intended as an approximation of the condition of being general type. We may
find an effective divisor $D\ge0$ in $\P(T_X)$ representing $\O(1)$, which is
locally a section of the bundle over $X$, although possibly not globally. Up to
finite degree issues, this gives a rational section of $\P(T_X)$, which as
mentioned earlier is precisely the data of a foliation on $X$, call it $\scrF$.

Now from $(df)^*\O(-1)=T_C$ we get $(df)^*\O(1)=K_C=2g(C)-2$, so if $C$ is
rational $C\cdot D<0$. But this forces $C$ to be a component of $D$, and so in
fact $\scrF$ is tangent to $C$. % why?
In other words, from the assumption that $\O(1)$ is big we have produced a
foliation on $X$ which is tangent to every rational curve in $X$. The existence
of such a foliation is a strong constraint, and this argument is part of work
due to Bogomolov and McQuillan \cite{mcquillan_98} which lead to certain
restricted results related to the conjecture.

\subsection{Ample vector bundles}

We would like to better understand this condition of $\O_{\P(T_X)}(1)$ being
big in terms of $X$ and $T_X$ directly. It turns out to be equivalent to a
notion of ``ample'' for the vector bundle $T_X$.

\begin{proposition}[\cite{hartshorne_66}, 3.2] \label{prop:ample}
    Let $\calE$ be a vector bundle on a variety $X$. Then $\calE$ is an ample
    vector bundle on $X$ iff the tautological line bundle $\O_{\P(\calE)}(1)$ on
    $\P(\calE)$ is ample.
\end{proposition}

We recall from \cite{hartshorne_66} the notion of an ample vector bundle:

\begin{definition}
    We say a vector bundle $\calE$ over a variety $X$ is \emph{ample} if for any
    coherent sheaf $\calF$ the sheaf $\calF\otimes\Sym^n\calE$ is generated by
    global sections for $n\gg0$.
\end{definition}

The key point to proving \cref{prop:ample} is the following fact:

\begin{lemma}
    If $p:\P(\calE)\to X$ is the projection map, then for every coherent sheaf
    $\calF$ on $X$ we have
    \begin{equation*}
        \dR p_*(p^*\calF\otimes\O_{\P(\calE)}(n))
            = \calF\otimes\Sym^n(\calE).
    \end{equation*}
\end{lemma}

\begin{proof}
    By the projection formula we have
    \begin{equation*}
        \dR p_*(p^*\calF\otimes\O_{\P(\calE)}(n))
            = \calF\otimes^\dL\dR p_*\O_{\P(\calE)}(n),
    \end{equation*}
    and $\dR p_*\O_{\P(\calE)}(n)=\Sym^n\calE$ as a relative version of
    \begin{equation*}
        H^k(\O_{\P(V)}(n)) = \begin{dcases*}
            \Sym^nV & $k=0$, $n\ge0$, \\ % is there a dual missing?
            0 & otherwise.
        \end{dcases*}
    \end{equation*}
\end{proof}

Motivated by the construction in the previous section, suppose we want to
construct a general type surface $X$ such that $\O_{\P(T_X)}(1)$ is big. From
\cref{prop:ample}, this is equivalent to requiring that $T_X$ is ample. We will
consider a degree $d$ hypersurface in $\P^3$. There is the normal bundle exact
sequence
\begin{equation*}
    0 \to N_{X/\P^3}^\vee \to \Omega^1_{\P^3}|_X \to \Omega^1_X \to 0,
\end{equation*}
and $N_{X/\P^3}=\O(-d)$ by adjunction. Taking determinants we see that
$\O(-4)=K_{\P^3}=K_X\otimes\O(-d)$, so at least $\det\Omega^1_X=K_X$ is ample
for $d\gg0$. It is not true that a vector bundle with ample determinant bundle
is always itself ample (although the converse holds). However, in this
particular case it is sufficient, due to the following result:

% TODO: counterexample
% TODO: explicit example with foliation?

\begin{proposition}\label{prop:stable ample}
    If $\calE$ is a slope semi-stable vector bundle, and $\det\calE$ is ample,
    then $\calE$ is ample.
\end{proposition}

To understand this proposition, we will move on to study slope stability in the
next section.

\begin{remark}
    To see that $\det\calE$ is ample whenever $\calE$ is ample, note that
    $\Sym^n\calE$ is ample and generated by global sections for $n\gg0$.
    Tensoring with a globally generated sheaf preserves ampleness, so
    $(\Sym^n\calE)^{\otimes r}$ is ample, and hence the quotient
    $\wedge^r\Sym^n\calE=\det(\Sym^n\calE)$ is also ample. But this line bundle
    is acted on by $\GL(\calE)$, and the only characters of $\GL$ are powers of
    the determinant, so $\det(\Sym^n\calE)=(\det\calE)^{\otimes m}$ for some
    $m$. Hence $\det\calE$ is ample.
\end{remark}

\subsection{Slope stability}

The notion of stability is motivated by the application of Mumford's geometric
invariant theory in the construction of moduli spaces of sheaves.

\begin{definition}
    Suppose $X$ is a complex projective variety, with ample line bundle $\O(1)$.
    For a coherent sheaf $\calE$ of pure dimension on $X$, we define the
    \emph{slope} of $\calE$ to be
    \begin{equation*}
        \mu(\calE) = \deg\calE / \rank\calE,
    \end{equation*}
    where $\deg\calE/(\dim X)!$ is the leading coefficient of the Hilbert
    polynomial $m\mapsto\chi(\calE\otimes\O(m))$. We say $\calE$ is slope
    \emph{semistable} if for any subsheaf $\calF\subset\calE$ we have
    $\mu(\calF)\le\mu(\calE)$, and \emph{stable} if the inequality is strict for
    proper subsheaves. A direct sum of stable sheaves of the same slope is
    called \emph{polystable}.
\end{definition}

We will only be considering slope stability, and so we drop the slope prefix and
write simply ``$\calE$ is (semi/poly)stable''. A key theorem in the theory of
slope stability is the following theorem, formerly the Calabi conjecture.

\begin{theorem}[Donaldson--Uhlenbeck--Yau] % TODO: reference
    If $X$ is a smooth complex projective manifold, then a holomorphic vector
    bundle $\calE$ on $X$ is slope polystable iff it admits a Hermite--Einstein
    metric.
    % irreducible vector bundles => can drop poly
\end{theorem}

To apply this in the situation above, we will appeal to two major theorems:

% TODO: introduce kahler einstein metrics

\begin{theorem}[Aubin, Yau]
    For a general type smooth complex projective variety, the tangent bundle
    admits a K\"ahler--Einstein metric.
\end{theorem}

\begin{theorem}[Kobayashi, L\"ubke]
    If the tangent bundle of a smooth complex projective variety admits a
    K\"ahler--Einstein metric, then it is slope stable.
\end{theorem}

% TODO: reference
Note that stability of a vector bundle is equivalence to stability of its dual),
so these results imply stability of $\Omega^1_X$, allowing us to apply
\cref{prop:stable ample}.

We now turn to look at some basic properties of slope stability.

\begin{proposition}
    If $0\to\calE'\to\calE\to\calE''\to0$ is a short exact sequence of vector
    bundles, then
    \begin{equation*}
        \min\{\mu(\calE'),\mu(\calE'')\}
            \le \mu(\calE) \le \max\{\mu(\calE'),\mu(\calE'')\}.
    \end{equation*}
\end{proposition}

\begin{lemma}
    For a vector bundle $V$, we have $V^\vee=V\otimes(\det V)^\vee$.
\end{lemma}

\begin{proposition}
    A vector bundle $\calE$ is (semi)stable iff $\calE^\vee$ is (semi)stable.
\end{proposition}

\begin{proposition}
    If $\calE,\calE'$ are semistable vector bundles, then so is
    $\calE\otimes\calE'$.
\end{proposition}

\begin{proposition}
    If $\calE,\calE'$ are semistable vector bundles, then so is
    $\calE\oplus\calE'$.
\end{proposition}
