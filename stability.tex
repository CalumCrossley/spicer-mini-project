
% TODO: in this section we blah

\subsection{The Green-Griffiths conjecture}

% TODO: intro / context

\begin{conjecture}[Green--Griffiths, \cite{griffiths_80}]
    Let $X$ be a smooth complex variety of general type. Then $X$ admits no
    Zariski dense holomorphic curves.
\end{conjecture}

% TODO: clean this up
Blah ideas due to Bogomolov, McQuillan. We consider a holomorphic map
$f:\C\to X$, which induces also a map $df:\C\to\P(T_X)$. Writing $\O(-1)$ for
the relative tautological bundle on $\P(T_X)$, we have (tautologically)
$(df)^*\O(-1)=T_C$. Now suppose $\O(1)$ is big. In fact this is equivalent to
the vector bundle $T_X$ being ample in a suitable sense which we will look at
later. % TODO: T_X vs Omega^1_X stuff
This assumption is intended as an approximation of the condition of being
general type. We may find an effective divisor $D\ge0$ in $\P(T_X)$ representing
$\O(1)$, which is locally a section of the bundle over $X$, although possibly
not globally. For simplicity, we assume that it is a section, which can be
ensured by a suitable base-change. Then we have a section of $\P(T_X)$, which as
mentioned earlier is precisely a foliation, say $\scrF$, on $X$.

Now from $(df)^*\O(-1)=T_C$ we get $(df)^*\O(1)=K_C=2g(C)-2$, so if $C$ is
rational $C\cdot D<0$. But this forces $C$ to be a component of $D$, and so in
fact $\scrF$ is tangent to $C$. % why?
In other words, from the assumption that $\O(1)$ is big we have produced a
foliation on $X$ which is tangent to every rational curve in $X$. The existence
of such a foliation is a strong constraint, and this argument is part of work
due to Bogomolov and McQuillan \cite{mcquillan_98} which lead to certain
restricted results related to the conjecture.

It turns out that the condition of $\O_{\P(T_X)}(1)$ being big as a property of
the vector bundle $T_X$ is equivalent to a notion of ampleness for $T_X$, as
defined in \cite{hartshorne_66}. % TODO: details / sketch

\subsection{Ampleness and slope stability}

% TODO: figure out T_X vs Omega^1_X

Motivated by the previous arguments, suppose we want an example of a surface $X$
such that $\O_{\P(T_X)}(1)$ is big, or equivalently $T_X$ is ample. For a
somewhat flexible class of general type surfaces, we will look among high degree
hypersurfaces in $\P^3$. Consider $X$ cut out by a section of $\O_{\P^3}(d)$. We
have the normal bundle exact sequence
\begin{equation*}
    0 \to N_{X/\P^3}^\vee \to \Omega^1_{\P^3}|_X \to \Omega^1_X \to 0,
\end{equation*}
where $N_{X/\P^3}=\O(-d)$ by adjunction. Taking determinants, we see that
$\O(-4)=K_{\P^3}=K_X\otimes\O(-d)$, so at least $\det\Omega^1_X=K_X$ is ample
for $d\gg0$. It is not true that a vector bundle is ample whenever its
determinant bundle is ample, although the converse does hold. % TODO: example, proof
But in this particular case it is actually sufficient, due to the following
result.

% TODO: prove this somewhere
\begin{proposition}
    If $\calE$ is a vector bundle such that $\det\calE$ is ample, and $\calE$ is
    semi-stable, then $\calE$ is ample.
\end{proposition}

% TODO: mention Omega^1_X being semi-stable because of Kahler-Einstein metrics
% easier proof?

% TODO: construct example, find the foliation?

Recall the notion of slope stability:

\begin{definition}
    We define the \emph{slope} of a coherent sheaf $\calE$ to be
    $\mu(\calE)\coloneqq\deg\calE/\rank\calE$. The sheaf is (slope)
    \emph{stable} if for all subsheaves $0\to\calF\to\calE$ we have
    $\mu(\calF)\le\calE$, and (slope) \emph{semi-stable} if the same holds but
    with a non-strict inequality. % TODO: check this
\end{definition}

This definition is motivated by the application of geometric invariant theory in
the construction of moduli spaces of sheaves. A key theorem in the stability
theory of sheaves is the following:

% TODO: K-E definition, reference, context
\begin{theorem}[Yau]
    A vector bundle $\calE$ is stable iff it admits a K\"ahler--Einstein metric.
\end{theorem}

\begin{proposition}
    If $\calE,\calE'$ are stable vector bundles, then so is $\calE\otimes\calE'$.
\end{proposition}

\begin{proposition}
    If $\calE,\calE'$ are stable vector bundles, then so is $\calE\oplus\calE'$.
\end{proposition}

% TODO: prove one directly, the other is done via Kahler-Einstein metrics
