
It is clear that a foliation interacts with curves on the surface, and in fact
the intersection theory of the line bundles associated to the foliation can be
computed in terms of local invariants. In this section we introduce two of these
local invariants, and look at an example constraining a foliation using
$K_\scrF$.

\subsection{Tangency}

For a curve which is not $\scrF$-invariant, the local leaves of $\scrF$
intersect it non-degenerately, and we can ask about the multiplicity of these
intersections.

\begin{definition}
    Suppose $C$ is a curve, and $p\in C$. We define the \emph{tangency order} of
    $\scrF$ to $C$ at $p$ as
    \begin{equation*}
        \tang(\scrF,C,p) = \dim_\C\O_{X,p}/(v(f),f),
    \end{equation*}
    where $f$ is a local equation defining $C$ and $v$ is the vector field
    generating $\scrF$, viewed as a derivation. We may then define the total
    tangency order of $\scrF$ along $C$:
    \begin{equation*}
        \tang(\scrF,C) = \sum_{p\in C}\tang(\scrF,C,p).
    \end{equation*}
\end{definition}

\begin{remark}
    The component of $C$ through $p$ is $\scrF$-invariant iff $v(f)$ vanishes
    along it, i.e. is a multiple of $f$, and so $\tang(\scrF,C,p)<\infty$ iff
    this is not the case. If $p\notin\Sing(\scrF)\cup\Sing(C)$ then
    $\tang(\scrF,C,p)=0$ if $\scrF$ and $C$ are transverse at $p$, so the sum
    defining $\tang(\scrF,C)$ is a finite sum provided $C$ has no
    $\scrF$-invariant components.
\end{remark}

\begin{proposition}\label{prop:tangency}
    Suppose $C$ has no $\scrF$-invariant components. Then
    \begin{equation*}
        T_\scrF\cdot C = \tang(\scrF,C) - C\cdot C.
    \end{equation*}
\end{proposition}

\begin{proof}
    $v(f)$ defines a holomorphic section of $K_\scrF\otimes\O(C)|_C$.
    % TODO
\end{proof}

% example

\subsection{Vanishing order}

For a curve $C$ which is $\scrF$-invariant, the 1-form defining $\scrF$ vanishes
along the curve, and we can ask about the order of this vanishing. Suppose $f$
is a local equation defining $C$, and $\omega$ is the 1-form defining $\scrF$
near $p$. Since $\omega$ vanishes on $C$, we can write
$g\cdot\omega=h\cdot df+f\cdot\eta$ for some holomorphic functions $g,h$ and
1-form $\eta$, where $h$ may be assumed coprime to $f$, \cite{neto_86}. (Roughly
speaking this comes from $\Omega^1_C=\O_X/(f)\otimes\Omega^1_X/(df)$.) The
meromorphic function $(h/g)|_C$ is in fact independent of this decomposition,
being the residue of $\frac{\omega}{f}=\frac{h}{g}\frac{df}{f}+\frac{\eta}{g}$
along $C$.

\begin{definition}
    We define the \emph{vanishing index} of $\scrF$ along $C$ at $p$ as
    \begin{equation*}
        Z(\scrF,C,p) = \ord_p\biggl(\frac{h}{g}\biggr)\bigg|_C,
    \end{equation*}
    which is independent of the choice of defining equation $f$ and 1-form
    $\omega$ since multiplication by a non-vanishing holomorphic function does
    not affect vanishing order. We may then define the total vanishing index of
    $\scrF$ along $C$:
    \begin{equation*}
        Z(\scrF,C) = \sum_{p\in C}Z(\scrF,C,p).
    \end{equation*}
\end{definition}

\begin{remark}
    If $p\notin\Sing(\scrF)$, then by \cref{thm:frobenius} we may assume
    $\omega=df$, so $h/g=1$ and $Z(\scrF,C,p)=0$. So really this index is an
    invariant of separatrices. This shows that the sum defining $Z(\scrF,C)$ is
    finite.
\end{remark}

\begin{remark}
    The meromorphic function $h/g$ may actually have a pole at $p$, so that
    $Z(\scrF,C,p)$ is negative. Consider the foliation given by
    $\omega=7ydx-4xdy$, with separatrix cut out by $f=x^{7}-y^{4}$. Then
    \begin{align*}
        \omega = \frac{xy\cdot df - 4xdy\cdot f}{x^4},
    \end{align*}
    so the index is the order of vanishing of $y/x^{4-1}$ on $C=\{f=0\}$ at
    $p=(0,0)$. From the normalization $t\mapsto(t^{4},t^{7})$ we compute
    $Z(\scrF,C,p)=7-4(4-1)=-5$. Fortunately this only occurs for singularities
    with infinitely many separatrices (necessarily non-reduced), and when $C$ is
    smooth we get agreement with the Poincar\'e--Hopf index at $p$ of the vector
    field generating the foliation; \cite{brunella_97}.
\end{remark}

\begin{proposition}
    Suppose $C$ is an $\scrF$-invariant curve. Then
    \begin{equation*}
        N_\scrF\cdot C = Z(\scrF,C) + C\cdot C.
    \end{equation*}
\end{proposition}

\begin{proof}
    $(h/g)|_C$ defines a meromorphic section of $N_\scrF\otimes\O(-C)|_C$.
    % TODO
\end{proof}

\subsection{A rational example}\label{sec:pencil}

Suppose $\scrF$ is a foliation of $\P^2$, and suppose the tangency order
$\tang(\scrF,\ell)$ is $d$ for a general line $\ell$. Then \cref{prop:tangency}
gives
\begin{equation*}
    K_\scrF\cdot\ell = d - \ell\cdot\ell = d - 1, \tag{$*$}
\end{equation*}
so $K_\scrF=\O(d-1)$. There is a natural tricohotomy between the case $d=0$,
where $K_\scrF$ is negative, the case $d=1$, where $K_\scrF$ is trivial, and the
general case $d\ge2$. We will consider the case $d=0$.

Specifically, we are assuming that the general line $\ell$ is transverse to
$\scrF$. In fact, by the formula ($*$) applying \cref{prop:tangency} again we
have $\tang(\scrF,\ell)=0$ for \emph{any} non-$\scrF$-invariant line $\ell$, so
all non-$\scrF$-invariant lines are transverse to $\scrF$. Now at any point in
$\P^2$ which is not a singularity of $\scrF$, there is a line through the point
sharing the tangent direction of $\scrF$, which is by construction not
transverse to $\scrF$. This line must be $\scrF$-invariant, and so we see that
$\scrF$ is a pencil of lines in $\P^2$.

So we have shown, a foliation of $\P^2$ with $K_\scrF$ negative must be a pencil
of lines (like \cref{ex:parallel}).
