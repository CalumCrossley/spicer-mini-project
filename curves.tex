
It is clear that a foliation interacts with curves on the surface, and in fact
the intersection theory of the line bundles associated to the foliation can be
computed in terms of local invariants. In this section we introduce two of these
local invariants, and look at an example constraining a foliation using
$K_\scrF$.

\subsection{Tangency}

For a curve which is not $\scrF$-invariant, the local leaves of $\scrF$
intersect it non-degenerately, and we can ask about the multiplicity of these
intersections.

\begin{definition}
    Suppose $C$ is a curve, and $p\in C$. We define the \emph{tangency order} of
    $\scrF$ to $C$ at $p$ as
    \begin{equation*}
        \tang(\scrF,C,p) = \dim_\C\O_{X,p}/(v(f),f),
    \end{equation*}
    where $f$ is a local equation defining $C$ and $v$ is the vector field
    generating $\scrF$, viewed as a derivation. We may then define the total
    tangency order of $\scrF$ along $C$:
    \begin{equation*}
        \tang(\scrF,C) = \sum_{p\in C}\tang(\scrF,C,p).
    \end{equation*}
\end{definition}

\begin{remark}
    The component of $C$ through $p$ is $\scrF$-invariant iff $v(f)$ vanishes
    along it, i.e. is a multiple of $f$, and so $\tang(\scrF,C,p)<\infty$ iff
    this is not the case. If $p\notin\Sing(\scrF)\cup\Sing(C)$ then
    $\tang(\scrF,C,p)=0$ if $\scrF$ and $C$ are transverse at $p$, so the sum
    defining $\tang(\scrF,C)$ is a finite sum provided $C$ has no
    $\scrF$-invariant components.
\end{remark}

\begin{proposition}\label{prop:tangency}
    Suppose $C$ has no $\scrF$-invariant components. Then
    \begin{equation*}
        T_\scrF\cdot C = \tang(\scrF,C) - C\cdot C.
    \end{equation*}
\end{proposition}

\begin{proof}
    $v(f)$ defines a holomorphic section of $K_\scrF\otimes\O(C)|_C$.
    % TODO
\end{proof}

% example

\subsection{Vanishing order}

For a curve which is $\scrF$-invariant, the 1-form defining $\scrF$ vanishes
along the curve, and we can ask about the order of this vanishing.

\begin{definition}
    Suppose $C$ is an $\scrF$-invariant curve, and $p\in C$. We define the
    \emph{vanishing order} of $\scrF$ along $C$ at $p$ as
    \begin{equation*}
        Z(\scrF,C,p) = \ord_p\biggl(\frac{h}{g}\biggr)\bigg|_C,
    \end{equation*}
    where blah. We may then define the total vanishing order of $\scrF$ along
    $C$:
    \begin{equation*}
        Z(\scrF,C) = \sum_{p\in C}Z(\scrF,C,p).
    \end{equation*}
\end{definition}

% smooth implies zero, so finite sum

% negative example!

% other example

\begin{proposition}
    Suppose $C$ is an $\scrF$-invariant curve. Then
    \begin{equation*}
        N_\scrF\cdot C = Z(\scrF,C) + C\cdot C.
    \end{equation*}
\end{proposition}

\begin{proof}
    $(h/g)|_C$ defines a meromorphic section of $N_\scrF\otimes\O(-C)|_C$.
    % TODO
\end{proof}

% where does this go?
% K_X.C+C^2 = -chi(C) euler characteristic of smoothing

\subsection{A rational example}\label{sec:pencil}

Suppose $\scrF$ is a foliation of $\P^2$, and suppose the tangency order
$\tang(\scrF,\ell)$ is $d$ for a general line $\ell$. Then \cref{prop:tangency}
gives
\begin{equation*}
    K_\scrF\cdot\ell = d - \ell\cdot\ell = d - 1, \tag{$*$}
\end{equation*}
so $K_\scrF=\O(d-1)$. There is a natural tricohotomy between the case $d=0$,
where $K_\scrF$ is negative, the case $d=1$, where $K_\scrF$ is trivial, and the
general case $d\ge2$. We will consider the case $d=0$.

Specifically, we are assuming that the general line $\ell$ is transverse to
$\scrF$. In fact, by the formula ($*$) applying \cref{prop:tangency} again we
have $\tang(\scrF,\ell)=0$ for \emph{any} non-$\scrF$-invariant line $\ell$, so
all non-$\scrF$-invariant lines are transverse to $\scrF$. Now at any point in
$\P^2$ which is not a singularity of $\scrF$, there is a line through the point
sharing the tangent direction of $\scrF$, which is by construction not
transverse to $\scrF$. This line must be $\scrF$-invariant, and so we see that
$\scrF$ is a pencil of lines in $\P^2$.

So we have shown, a foliation of $\P^2$ with $K_\scrF$ negative must be a pencil
of lines (like \cref{ex:parallel}).
