
\section{Introduction}

% TODO: write introduction

Main reference: \cite{brunella_book}.

\section{Foliations and blowups}

\subsection{Foliations}

% assume normal surfaces (or smooth even (check where necessary?))

% motivation?, definition, bundles involved, singular locus, invariant curves, separatrices, leaves

% TODO: extension problem needs sorting out
The map of vector bundles $T_\scrF\to T_X$ has cokernel which is a vector bundle
away from the locus where the relevant minors vanish. By assumption this locus
has codimension at least 2, so this vector bundle extends uniquely to a vector
bundle $N_\scrF$ on $X$, such that the cokernel is $\calI_\scrF\cdot N_\scrF$
where $\calI_\scrF$ is the ideal sheaf generated by these matrix minors. In the
surface case, this is the vanishing locus of the associated vector field.

\subsection{Holonomy}

\subsection{Blowups}

In view of the characterization of birational morphisms as sequences of blowups,
we would like to understand how foliations interact with blowups.

% TODO: double check these

\begin{example}
    Consider the foliation given by $\omega=xdy-ydx$ on $\A^2$. % TODO: tikz

    To better deal with all these curves through the singularity, we would like
    to blow up the origin. Write $X=\Bl_{(0,0)}\A^2=U\cup V$, where
    $U=\{(x,tx,1:t)\in\A^2\times\P^1\}$, $V=\{(sy,y,s:1)\in\A^2\times\P^1\}$.
    The pullbacks to $U$ and $V$ of $\omega$ are
    \begin{equation*}
        \omega|_U = xd(tx)-txdx = x^2dt, \quad
        \omega|_V = sydy - yd(sy) = -y^2ds.
    \end{equation*}
    These vanish to second order along the exceptional divisor $E$, but
    factoring this out we get 1-forms $dt$ and $-ds$ on $U$ and $V$ which patch
    together to give a nowhere vanishing 1-form with values in the line bundle
    $\O(2E)$. (Note that $dt=-ds/s^2$, $-ds=dt/t^2$.) Pictorially this
    is exactly as expected. % TODO: tikz
\end{example}

\begin{example}
    If we instead have the singularity $\omega=xdy+ydx$, then
    \begin{equation*}
        \omega|_U = x^2dt + 2xtdx, \quad
        \omega|_V = y^2ds+2ysdy.
    \end{equation*}
    This has only first order vanishing along $E$, giving a 1-form valued in
    $\O(E)$, but with two singularities from the local forms $xdt+2tdx$,
    $yds+2sdy$. The exceptional divisor takes the role of the other separatrix
    downstairs in lifting the singularity.
\end{example}

\begin{example}\label{ex:smooth blowup}
    Finally, consider blowing up a smooth foliation; $\omega=dx$. Then
    \begin{equation*}
        \omega|_U = dx, \quad \omega|_V = sdy+yds,
    \end{equation*}
    so we have introduced a singularity on the exceptional divisor, which
    crosses the strict transform of the unique separatrix downstairs. Note that
    the pullback doesn't vanish along $E$.
\end{example}

\begin{definition}
    Suppose $(X,\scrF)$ is a foliated surface, and $p\in X$. Under the blowup
    $\pi:\tilde X=\Bl_pX\to X$ we get a rational section of
    $\pi^*N_\scrF\otimes\Omega^1_{\tilde X}$, with a pole of order $a\ge0$ on
    $E$. This then gives a section of
    $\pi^*N_\scrF(aE)\otimes\Omega^1_{\tilde X}$ defining $\tilde\scrF$, a
    foliation on $\tilde X$ with $N_{\tilde\scrF}=\pi^*N_\scrF(aE)$. Write
    $a(\scrF,p)\coloneqq a$.
\end{definition}

\begin{remark}
    The same reasoning applies to pullback a foliation along any birational
    morphism, adding exceptional divisors to the normal bundle, and more
    generally we can transport a foliation along any rational map, albeit with
    less control of the normal bundle. This is clear by using the above logic to
    identify foliations as rational sections of $\Omega^1_X$ (or $T_X$) modulo
    multiplication by rational sections of $\O_X$.
\end{remark}

\begin{remark}
    By \cref{ex:smooth blowup}, we have $a(\scrF,p)=0$ for
    $p\notin\Sing(\scrF)$.
\end{remark}

\section{Reduced singularities}

\section{Numerical properties}

Waffle about canonical bundle.

Tangency orders, vanishing orders, intersection numbers.

\section{Rational fibrations}

\subsection{Rational surfaces}

\subsection{Miyaoka's criterion}

Consider a fibration $f:X\to C$ where the general fiber of $f$ is rational. For
example $X=\P^1\times\P^1$ with the projection to the first coordinate.
$K_\scrF=K_{X/C}-\{\text{effective supported on fibers}\}$,
$K_\scrF\cdot F=K_X\cdot F=-2$ for fiber $F$, $F$ nef, so for $H$ ample $nF+H$
also ample, with $K_\scrF\cdot(nF+H)=-2n+K_\scrF\cdot H<0$ for $n\gg0$.

\begin{theorem}[Miyaoka]
    Suppose $(X,\scrF)$ is a foliated surface, and $H$ is an ample divisor on
    $X$ such that $K_\scrF\cdot H<0$. Then there is a birational morphism
    $\tilde X\to X$ and a fibration $f:\tilde X\to C$ such that the general
    fiber of $f$ is rational, and $\scrF$ is the tame fibration induced by $f$.
\end{theorem}

\begin{proof}[Proof (Bogomolov, McQuillan)]
    Bertini theorem $\implies$ find $C\in|nH|$ smooth disjoint from
    $\Sing(\scrF)$. Let $Y=X\times C$. Consider the rank 1 foliation $\scrG$ on
    $Y$ given by $\scrG|_{X\times\{c\}}=\scrF$. Consider the diagonal
    $\Delta\subset C\times C\subset X\times C=Y$, which is disjoint from
    $\Sing(\scrG)$. Take the integral curves of $\scrG$ through $\Delta$ in a
    neighbourhood of $\Delta$ to define an analytic surface $S$ through
    $\Delta$. Then $N_{\Delta/S}\simeq T_\scrG|_\Delta\simeq T_\scrF|_C$, so
    $\deg N_{\Delta/S}=T_\scrF\cdot C=nT_\scrF\cdot H>0$. Hence $N_{\Delta/S}$
    is ample. It follows that $S$ is algebraic (Voisin 1st lecture). Let
    $\Sigma$ be the Zariski closure of $S$.
\end{proof}
