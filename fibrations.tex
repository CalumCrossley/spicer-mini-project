
In this section we look at a few examples of foliations arising from fibrations.

\subsection{Algebraic integrability}

It is a natural question to ask when the leaves of a foliation are algebraic
curves. Examples we have seen are $xdy-ydx$, with leaves $ax+by=0$ for
$[a:b]\in\P^1$, and $pxdy+qydx$ for $p,q\in\N$, with leaves $x^qy^p=c$ for
$c\in\A^1$. Notice that $xdy-ydx=y\cdot d(x/y)$, and
$pxdy+qydx=d(x^qy^p)/(x^{q-1}y^{p-1})$, so these are examples of the following
construction.

\begin{definition}
    Given a rational map $f:X\dashrightarrow C$, where $X$ is a surface and $C$
    is a curve, the relative tangent bundle $T_{X/C}$ defines a foliation on $X$
    which we call a \emph{fibration}. The leaves of this fibration are precisely
    the fibers of $f$, which are algebraic curves.
\end{definition}

% TODO: compute N_F = K_C + Σ(m-1)D where fibers of f are ΣmD

In the converse direction, there is a general statement which can be made:

\begin{proposition}
    Suppose $(X,\scrF)$ is a foliated surface, such that for a general $x\in X$
    the leaf of $\scrF$ through $x$ is algebraic. Then there exists a rational
    map $X\dashrightarrow Y$ such that $T_{X/Y}$ and $T_\scrF$ are isomorphic on
    a dense open subset of $X$.
\end{proposition}

\begin{proof}[Sketch proof]
    We consider the Hilbert scheme $\Hilb_X$. Tangency to the foliation $\scrF$
    is a condition on subvarieties which cuts out a closed subscheme
    $\calT_{X,\scrF}\subseteq\Hilb_X$. By assumption, the canonical map from the
    universal family $\calU$ over $\calT_{X,\scrF}$ to $X$ is dominant. Since
    $\calU$ is an infinite disjoint union of projective components, by the Baire
    category theorem we have some such component $Z\subset\calU$ over
    $Y\subset\calT_{X,\scrF}$ which maps surjectively to $X$. After restricting
    to a suitable hyperplane $Y^0$ in $Y$, we get a finite surjection
    $Z^0\to X$, which lifts to a Galois covering $\tilde Z^0\to X$, where the
    Galois group $G$ can be made to act on a base $\tilde Y^0$.
    \begin{equation*}
        \begin{tikzcd}
            \tilde Z^0 \ar[d] \ar[r]
                \ar[rrr,bend left,"\text{Galois}"] &
            Z^0 \ar[d] \ar[r,hook] &
            \calU \ar[d] \ar[r] & X \\
            \tilde Y^0 \ar[r] & Y^0 \ar[r,hook] & \calT_X
        \end{tikzcd}
    \end{equation*}
    At this point we can collapse the fibers by taking the quotient
    $\tilde Z^0/G$, which maps birationally to $X$, and so we get a rational map
    $X\dashrightarrow \tilde Y^0/G$ which by construction has fibers tangent to
    $\scrF$.
\end{proof}

% TODO expand on Galois coverings, expand on hyperplane section argument

\subsection{Miyaoka's criterion}

We saw in \cref{sec:pencil} an example where $K_\scrF$ being negative forces the
foliation $\scrF$ to consist of rational curves. We present a theorem due to
Miyaoka which asserts a more general result of this form.

% condition = K_F pseff?
\begin{theorem}[Miyaoka]
    Suppose $(X,\scrF)$ is a foliated surface, and $H$ is an ample divisor on
    $X$ such that $K_\scrF\cdot H<0$. Then there is a birational morphism
    $\tilde X\to X$ and a fibration $f:\tilde X\to C$ such that the general
    fiber of $f$ is rational, and $\scrF$ is the foliation induced by $f$.
\end{theorem}

\begin{remark}
    % TODO: cleanup
    % necessary
    Consider a fibration $f:X\to C$ where the general fiber of $f$ is rational. For
    example $X=\P^1\times\P^1$ with the projection to the first coordinate.
    $K_\scrF=K_{X/C}-\{\text{effective supported on fibers}\}$,
    $K_\scrF\cdot F=K_X\cdot F=-2$ for fiber $F$, $F$ nef, so for $H$ ample $nF+H$
    also ample, with $K_\scrF\cdot(nF+H)=-2n+K_\scrF\cdot H<0$ for $n\gg0$.
\end{remark}

\begin{proof}[Proof (Bogomolov, McQuillan)]
    Bertini theorem $\implies$ find $C\in|nH|$ smooth disjoint from
    $\Sing(\scrF)$. Let $Y=X\times C$. Consider the rank 1 foliation $\scrG$ on
    $Y$ given by $\scrG|_{X\times\{c\}}=\scrF$. Consider the diagonal
    $\Delta\subset C\times C\subset X\times C=Y$, which is disjoint from
    $\Sing(\scrG)$. Take the integral curves of $\scrG$ through $\Delta$ in a
    neighbourhood of $\Delta$ to define an analytic surface $S$ through
    $\Delta$. Then $N_{\Delta/S}\simeq T_\scrG|_\Delta\simeq T_\scrF|_C$, so
    $\deg N_{\Delta/S}=T_\scrF\cdot C=nT_\scrF\cdot H>0$. Hence $N_{\Delta/S}$
    is ample. It follows that $S$ is algebraic (Floris 1st lecture). Let
    $\Sigma$ be the Zariski closure of $S$. % TODO: complete this (from Floris)
\end{proof}
